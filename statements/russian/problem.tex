\begin{problem}{3 точки}{стандартный ввод}{стандартный вывод}{1 секунда}{256 мегабайт}

Даны три целых числа $a, b $ и $c$ --- координаты точек на числовой прямой. За одну операцию можно выбрать упорядоченную пару точек, координату одной из них увеличить на 1, а координату другой уменьшить на 1. Иными словами, если у нас были две точки с координатами $u$ и $v$, мы выбрали пару $(u, v)$, то после операции у нас будут точки с координатами $u+1$ и $v-1$. Определите, возможно ли такими операциями сделать координаты всех точек равными, и если это возможно, то найдите минимальное количество операций за которое это можно сделать. В некоторых тестах, также необходимо найти последовательность операций позволяющих этого добиться.

\InputFile
Первая строка содержит целое число $t$ $( t = 0 \text{ или } t = 1)$. В случае, если $t = 0$ необходимо вывести только минимальное количество операций, а в случае, если $t = 1$ необходимо также вывести сами операции.

Вторая строка содержит три целых числа $a,b$ и $c$ --- изначальные координаты точек на числовой прямой $(|a|, |b|, |c| \leq 10^9, \text{ если } t = 0 \text{ и } |a|, |b|, |c| \leq 10^5,  \text{ если } t = 1)$. 

\OutputFile
В первой строке выведите Yes или No, в зависимости от того, можно ли сделать координаты всех точек равными.

Во второй строке выведите минимальное количество операций. 

Если $t = 1$, то в $(i+2)-$ ой строке выведите $u$ и $v$, если $i-$ая операция заключалась в выборе пары $(u,v)$.

Если возможных вариантов ответа несколько~--- выведите любой из них.

\Scoring
В этой задаче 20 тестов, не считая тестов из условия. Каждый тест оценивается независимо в 5 баллов.

Решения, верно работающие при $t = 0$, будут получать не менее 50 баллов.

\Examples

\begin{example}
\exmpfile{example.01}{example.01.a}%
\exmpfile{example.02}{example.02.a}%
\exmpfile{example.03}{example.03.a}%
\end{example}

\end{problem}

